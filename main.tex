\documentclass[11pt,A4]{article}
\usepackage[T1]{fontenc}
\usepackage[utf8]{inputenc}
\usepackage[UKenglish]{babel}
\usepackage{mathtools}
\usepackage{amsthm}
\usepackage{amssymb}
%\usepackage{mathrsfs}
\usepackage[mathscr]{euscript}
\usepackage{enumitem}
\usepackage{tikz-cd}
\usepackage{hyperref}
\usepackage[noabbrev]{cleveref}
\usepackage{todonotes}

% Plain theorems
\theoremstyle{plain}
\newtheorem{thm}{Theorem}[section]
\newtheorem{lm}[thm]{Lemma}
\newtheorem{prop}[thm]{Proposition}
\newtheorem{cor}[thm]{Corollary}

% Definition theorems
\theoremstyle{definition}
\newtheorem{defn}[thm]{Definition}
\newtheorem{exa}[thm]{Example}

% Remark theorems
\theoremstyle{remark}
\newtheorem{rem}[thm]{Remark}
\newtheorem{q}[thm]{Question}

% Mathbb for sets of numbers
\newcommand{\N}{\mathbb{N}}

% Mathscr for categories
\newcommand{\C}{\mathscr{C}}
\newcommand{\Top}{\mathscr{T}op}
\newcommand{\CHaus}{\mathscr{CH}aus}
\newcommand{\Ab}{\mathscr{A}b}
\newcommand{\Set}{\mathscr{S}et}
\newcommand{\D}{\mathscr{D}}

% Math operators
\DeclareMathOperator{\Hom}{Hom}
\DeclareMathOperator{\Cond}{Cond}
\DeclareMathOperator{\Ob}{Ob}

% New commands
\newcommand{\pe}{*_{pro\acute et}}
\renewcommand{\u}[1]{\underline{#1}}
\newcommand{\ot}{\otimes}
\newcommand{\op}{\oplus}
\newcommand{\fp}[1]{\times_{#1}}
\newcommand{\id}{\mathrm{id}}


\title{Seminar on Condensed Mathematics \\ \large Talk 2: Condensed Abelian Groups}
\author{Pedro Núñez}
\date{\today}

\begin{document}

\maketitle

\tableofcontents

\section{The category of condensed abelian groups}

Recall from the previous talk:

\begin{defn}
    The \textit{pro-étale site} of a point, denoted $*_{proét}$, consists of:
    \begin{itemize}
	\item The category whose objects are profinite sets $S$ (topological spaces homeomorphic to an inverse limit of finite discrete topological spaces, a.k.a. totally disconnected compact Hausdorff spaces) and whose morphisms are continuous maps.
	\item The coverings of a profinite set $S$ are all finite families of jointly surjective maps, i.e. all families of morphisms $\{ S_{i}\to S\}_{i\in I}$ indexed by finite sets $J$ such that $\sqcup_{i\in I}S_{i}\to S$ is surjective.
    \end{itemize}
\end{defn}

It is easy to check that the axioms of a covering family are satisfied (see \cite[\href{https://stacks.math.columbia.edu/tag/00VH}{Tag 00VH}]{sta19}), so we have a well-defined site.
Let us start by characterizing sheaves on this site:

\begin{lm}\label{lm:sheaf}
    A contravariant functor $T$ from $\pe$ to the category of abelian groups $\Ab$ is a sheaf if and only if the following three conditions hold:
    \begin{enumerate}[label=\alph*)]
	\item $T(\varnothing)=0$.
	\item For all profinite sets $S_{1}$ and $S_{2}$, the inclusions $i_{1}\colon S_{1}\to S_{1}\sqcup S_{2}$ and $i_{2}\colon S_{2}\to S_{1}\sqcup S_{2}$ induce a group isomorphism
	    \[ T(S_{1}\sqcup S_{2})\xrightarrow{T(i_{1})\times T(i_{2})} T(S_{1})\op T(S_{2}).\]
	\item Every surjection $f\colon S'\twoheadrightarrow S$ induces a group isomorphism
	    \[ T(S)\xrightarrow{T(f)} \{ g\in T(S') \mid T(p_{1})(g)=T(p_{2})(g) \in T(S'\times_{S}S') \}, \]
	    where $p_{1},p_{2}\colon S'\times_{S}S'\to S'$ denote the projections.
    \end{enumerate}
    \begin{proof}
	By definition, a functor $T$ is a sheaf on $\pe$ if and only if for every finite family $\{ S_{i}\to S\}_{i\in I}$ of jointly surjective morphisms the diagram
	\[ T(S)\to \prod_{i\in I} T(S_{i})\overset{p_{1}^{*}}{\underset{p_{2}^{*}}{\rightrightarrows}}\prod_{(i,j)\in I^{2}} T(S_{i}\times_{S} S_{j}) \]
	is exact, meaning that the left arrow is an equalizer of the two arrows on the right, which are explicitly described below.
	Since we are in $\Ab$ and $I$ is a finite set, we can reformulate this as the sequence
	\begin{equation}\label{eqn:sheaf}
	    0\to T(S)\to \bigoplus_{i\in I} T(S_{i})\xrightarrow{p_{1}^{*}-p_{2}^{*}} \bigoplus_{(i,j)\in I^{2}} T(S_{i}\times_{S}S_{j})
	\end{equation}
	being exact.

	So assume first that \ref{eqn:sheaf} is exact.
	The empty set is covered by the empty family, so the $T(\varnothing)$ must be a subgroup of $0$, hence $0$ itself.
	The natural inclusions $S_{i}\to S_{1}\sqcup S_{2}=S$ cover $S$ for $i\in I=\{1,2\}$, so it suffices to show that $T(p_{1})=T(p_{2})$ to verify $b)$.
	Since $i_{1}(S_{1})\cap i_{2}(S_{2})=\varnothing $, we have $S_{1}\times_{S}S_{2}=S_{2}\times_{S}S_{1}=\varnothing $.
	So if $(g,h)\in T(S_{1})\op T(S_{2})$, then
	\begin{multline*}
	    p_{1}^{*}(g,h)=(T(p_{1}^{S_{1}\times_{S}S_{1}})(g),T(p_{1}^{S_{1}\times_{S}S_{2}})(g),T(p_{1}^{S_{2}\times_{S}S_{1}})(h),T(p_{1}^{S_{2}\times_{S}S_{2}})(h)) \\ 
	    =(T(p_{1}^{S_{1}\times_{S}S_{1}})(g),0,0,T(p_{1}^{S_{2}\times_{S}S_{2}})(h)) \\
	    =(T(p_{2}^{S_{1}\times_{S}S_{1}})(g),T(p_{2}^{S_{1}\times_{S}S_{2}})(h),T(p_{2}^{S_{2}\times_{S}S_{1}})(g),T(p_{2}^{S_{2}\times_{S}S_{2}})(h))=p_{2}^{*}(g,h).
	\end{multline*}
	To show $c)$, note that $S'\twoheadrightarrow S$ is already a cover, so the exactness of \ref{eqn:sheaf} immediately implies the result.

	For the converse, suppose that $\{f_{j}\colon S_{j}\to S\}_{j\in J}$ is a collection of jointly surjective morphisms indexed by $J=\{ 1,\ldots,m\}$.
	We have then a surjection $f\colon S \twoheadrightarrow S$, where $S':=\sqcup_{j=1}^{m}S_{j}$ and $f:=\sqcup_{j=1}^{m}f_{j}$.
	By $c)$ we can write $T(S)=\{ g\in T(S')\mid T(p_{1}^{S'\fp{S}S'})(g)=T(p_{2}^{S'\fp{S}S'})(g)\}$.
	But $S'\fp{S}S'=\sqcup_{(a,b)\in J^{2}} S_{a}\fp{S}S_{b}$ and $p_{\varepsilon }^{S'\fp{S}S'}=\sqcup_{(a,b)\in J^{2}} i_{a}\circ p_{\varepsilon }^{S_{a}\fp{S}S_{b}}$ for $\varepsilon \in \{1,2\}$ and for $i_{a}\colon S_{a}\to S'$ the inclusions, so condition $b)$ allows us to rewrite this as
	\[ T(S)=\bigcap_{(a,b)\in J^{2}} \{ (g_{1},\ldots,g_{m})\in T(S_{1})\op \cdots \op T(S_{m})\mid P_{a,b}(g_{a},g_{b})\}\]
	with $P_{a,b}(g_{a},g_{b}):\equiv T(p_{1}^{S_{a}\fp{S}S_{b}})(g_{a})=T(p_{2}^{S_{a}\fp{S}S_{b}})(g_{b})$, which is what we wanted.	
    \end{proof}
\end{lm}

\begin{defn}
    A \textit{condensed abelian group} is a sheaf of abelian groups on $\pe$.
    We denote the category of condensed abelian groups by $\Cond(\Ab)$.
\end{defn}

From \cref{lm:sheaf} we deduce:

\begin{exa}
    Let $G$ be a topological abelian group. Then the functor $\u{G}=\Hom_{\Top}(-,G)\colon \pe\to \Ab$ is a condensed abelian group.
    Conditions $a)$ and $b)$ in \cref{lm:sheaf} are clear.
    For condition $c)$, let $f\colon S'\twoheadrightarrow S$ be a surjection.
    We want to show that $f^{*}\colon \u{G}(S)\to \u{G}(S')$ induces an isomorphism between the set of continuous maps $g\colon S\to G$ and the set of continous maps $h\colon S'\to G$ such that $h\circ p_{1}=h\circ p_{2}$.
    The image of $f^{*}$ is indeed contained in the set of all such maps, because $f\circ p_{1}=f\circ p_{2}$.
    Moreover, since $f$ is an epimorphism, $f^{*}$ is injective.
    So it only remains to show surjectivity.
    Let $h\colon S'\to G$ be a map such that $h\circ p_{1}=h\circ p_{2}$.
    For each $s\in S$, let $g(s):=h(s')$ for some $s'\in f^{-1}(\{s\})$.
    This is well-defined in $\Set$, because if $f(s')=f(s'')=s$, then $(s',s'')\in S'\times_{S}S'$, and thus we can write
    \[ g(s)=h(s')=h\circ p_{1}(s',s'')=h\circ p_{2}(s',s'')=h(s'').\]
    But in fact it is a morphism in $\Top$.
    Indeed, $S$ carries the quotient topology induced by $f$, so $f^{-1}(g^{-1}(U))=h^{-1}(U)$ being open in $S'$ for all $U$ open in $G$ implies that $g^{-1}(U)$ is open in $S$ for all $U$ open in $G$.
\end{exa}

Recall that a category $\C$ is called \textit{abelian} if it satisfies the following properties:
\begin{enumerate}[label=\roman*)]
    \item There exists a zero object $0\in \Ob(\C)$.
    \item For all $A,B\in \Ob(\C)$, their product $A\xleftarrow{p_{A}} A\times B\xrightarrow{p_{B}} B$ and their coproduct $A\xrightarrow{i_{A}}A\sqcup B\xleftarrow{i_{B}} B$ exist in $\C$ and the canonical morphism $A\sqcup B\xrightarrow{(\id_{A}\times 0) \sqcup (0\times \id_{B})} A\times B$ is an isomorphism\footnote{We identify $A\sqcup B$ with $A\times B$ via this isomorphism and call the result the \textit{direct sum} of $A$ and $B$, denoted $A\op B$.} in $\C$.
    \item Every morphisms has a kernel and a cokernel in $\C$.
    \item Every monomorphism in $\C$ is a kernel and every epimorphism in $\C$ is a cokernel.
\end{enumerate}

In particular, $\C$ is naturally an additive category and all finite limits and colimits exist in $\C$.

Recall also Grothendieck's axioms:
\begin{description}
    \item[(AB3)] All colimits exist.
    \item[(AB3*)] All limits exist.
    \item[(AB4)] (AB3) holds and all coproducts are exact.
    \item[(AB4*)] (AB3*) holds and all products are exact.
    \item[(AB5)] (AB3) holds and filtered colimits are exact.
    \item[(AB6)] (AB3) holds and for any family $\{ I_{j}\}_{j\in J}$ of filtered categories indexed by a set $J$ with functors $F_{j}\colon I_{j}\to \C$ the canonical morphism
	\[ \varinjlim_{(i_{j}\in I_{j})_{j}} \prod_{j\in J}F_{j}(i_{j})\to \prod_{j\in J}\varinjlim_{i_{j}\in I_{j}} F_{j}(i_{j}) \]
	is an isomorphism in $\C$.
\end{description}

\begin{rem}
    To check (AB3) on an abelian category it suffices to show that arbitrary coproducts exist, since we can build any colimit from coproducts and coequalizers, and similarly for the dual statement (AB3*).
    Colimits preserve colimits because the colimit functor is left adjoint to the diagonal functor.
    In particular coproducts are always right exact, so to check (AB4) on an abelian category satisfying (AB3) it suffices to check that coproducts preserve monomorphisms, and similarly for the dual statement (AB4*).
\end{rem}

The main goal of this talk is to show that the category $\Cond(\Ab)$ is an abelian category which verifies all the previous axioms.
Except for axioms (AB4*) and (AB6), this is a general fact which holds on any category of sheaves of abelian groups on a site, but we will still give an explicit proof in our situation.
This will be much easier after we express $\Cond(\Ab)$ as the category of sheaves of abelian groups on a simpler site.

\begin{defn}
    A compact Hausdorff\footnote{If we do not require Hausdorffness, an extremally disconnected space could be very connected, e.g. any irreducible topological space.} topological space is called \textit{extremally disconnected} if the closure of every open set is again open.
\end{defn}

Equivalently, a compact Hausdorff topological space is extremally disconnected if the closures of every pair of disjoint open sets are also disjoint.
Therefore extremally disconnected spaces are totally disconnected compact Hausdorff spaces, hence profinite spaces.
The converse is not true:

\begin{exa}
    In an extremally disconnected space every convergent sequence is eventually constant (see \cite[Theorem 1.3]{gle58}), so the $p$-adic integers are profinite but not extremally disconnected, because $(p^{n})_{n=1}^{\infty}$ converges to $0$.
\end{exa}
    
Extremally disconnected spaces are precisely the projective objects in the category $\CHaus$ of compact Hausdorff spaces (see \cite[Theorem 2.5]{gle58}).
Since pullbacks exist and preserve epimorphisms (i.e. surjective continuous functions\footnote{Urysohn's lemma implies that epimorphisms in the category of compact Hausdorff spaces are precisely surjective continuous functions, which is not true in the whole category of Hausdorff spaces (e.g. the inclusion of the complement of a point in the real line).}) in $\CHaus$, a compact Hausdorff space $S$ is extremally disconnected if and only if any surjection $S'\twoheadrightarrow S$ from a compact Hausdorff space admits a section:
\begin{center}
    \begin{tikzcd}
	& S'\arrow[twoheadrightarrow]{d} \\
	S\arrow[equal]{r}\arrow[dashed]{ur} & S
    \end{tikzcd}
\end{center}

The inclusion functor $U\colon \CHaus\to \Top$ has a left adjoint $\beta\colon \Top\to \CHaus$ called the \textit{Stone-\v{C}ech compactification}.
If $X$ is a discrete topological space, then $\beta X$ is extremally disconnected, because if $S\twoheadrightarrow \beta X$ is a continuous surjection from a compact Hausdorff topological space we may first lift the canonical map $X\to \beta X$ to $S$ and then extend it to a section from $\beta X$ by its universal property.
In particular, every compact Hausdorff space admits a surjection $\beta(X_{disc})\twoheadrightarrow X$ from an extremally disconnected space.
This has the following consequence:

\begin{prop}
    $\Cond(\Ab)$ is equvalent to the category of sheaves of abelian groups on the site of extremally disconnected spaces via the restriction from profinite sets.
    \begin{proof}
	Let $T\in \Cond(\Ab)$ and let $S$ be a profinite set.
	Let $f\colon \tilde{S}\twoheadrightarrow S$ and $g\colon \tilde{\tilde{S}}\twoheadrightarrow \tilde{S}\fp{S}\tilde{S}$ be continuous surjections from extremally disconnected spaces.
	By \cref{lm:sheaf} we have $T(S)=\ker(T(p_{1})-T(p_{2}))$, where $p_{1},p_{2}\colon \tilde{S}\fp{S}\tilde{S}$ denote the projections from the fiber product.
	Since $T(\tilde{S}\fp{S}\tilde{S})\xrightarrow{T(g)} T(\tilde{\tilde{S}})$ is injective, the kernel of $T(\tilde{S})\xrightarrow{T(p_{1})-T(p_{2})} T(\tilde{S}\fp{S}\tilde{S})$ is the same as the kernel of the composition $T(\tilde{S})\xrightarrow{T(p_{1}\circ g)-T(p_{2}\circ g)} T(\tilde{\tilde{S}})$.
	Therefore the value of $T$ at the profinite set $S$ is completely determined by the value of $T$ at the extremally disconnected sets $\tilde{S}$ and $\tilde{\tilde{S}}$.
    \end{proof}
\end{prop}

\begin{cor}\label{cor:sheaf}
    $\Cond(\Ab)$ is equivalent to the category of contravariant functors $T$ from the category of extremally disconnected sets to the category of abelian groups such that:
    \begin{enumerate}[label=\alph*)]
	\item $T(\varnothing)=0$.
	\item For all extremally disconnected sets $S_{1}$ and $S_{2}$, the inclusions $i_{1}\colon S_{1}\to S_{1}\sqcup S_{2}$ and $i_{2}\colon S_{2}\to S_{1}\sqcup S_{2}$ induce a group isomorphism
	    \[ T(S_{1}\sqcup S_{2})\xrightarrow{T(i_{1})\times T(i_{2})} T(S_{1})\op T(S_{2}).\]
    \end{enumerate}
    \begin{proof}
	We need to show that condition $c)$ in \cref{lm:sheaf} is automatically satisfied.
	So let $f\colon S'\twoheadrightarrow S$ be a continuous surjection of extremally disconnected spaces.
	Then we can find a section $\sigma\colon S\to S'$ with $f\circ \sigma =\id_{S}$.
	This implies that $T(\sigma)\circ T(f)=\id_{T(S)}$, so $T(f)$ is injective.
	The image of $T(f)$ is contained in $\{ g\in T(\tilde{S})\mid T(p_{1})(g)=T(p_{2})(g)\}$, because $f\circ p_{1}=f\circ p_{2}$.
	And conversely if $g\in T(\tilde{S})$ is such that $T(p_{1})(g)=T(p_{2})(g)$, then $T((\sigma \circ f)\fp{S}\id_{\tilde{S}})(T(p_{1})(g))=T((\sigma \circ f)\fp{S}\id_{\tilde{S}})(T(p_{2})(g))$, so $T(f)(T(\sigma)(g))=g$ and $g$ is in the image of $T(f)$.
    \end{proof}
\end{cor}

\section{$\infty$-categories}

\subsection{Motivation}

Let $X$ be a topological space.
We can try to consider this as a category in which the objects are the points of $X$ and the ($1$-)morphisms are the paths in $X$, which can be seen as homotopies between maps $\{ *\}\to X$.
We can consider a homotopy between paths a $2$-morphism, a homotopy between homotopies of paths a $3$-morphism, and so on.
We obtain an object called the \textit{fundamental $\infty$-grupoid} of $X$, denoted $\pi_{\leqslant \infty}(X)$.
Even if we just consider its objects and its $1$-morphisms, $\pi_{\leqslant \infty }(X)$ is not a category, because the composition of paths is not associative.
But it is associative up to a $2$-morphism, and similarly composition of $n$-morphisms is only associative up to an $n+1$-morphism.
Note also that $n$-morphisms are invertible up to an $n+1$-morphism for all $n\in \N_{>0}$.

\begin{q}
    How similar or different is $\pi_{\leqslant \infty}(X)$ from an usual category?
\end{q}

If $\C$ is any category, then we can think of it as having no $n$-morphisms for any $n\in \N_{>1}$.
Then associativity up to higher morphisms is still true in this case, so we can say that this is something that $\C$ and $\pi_{\leqslant \infty }(X)$ have in common.
On the other hand, the $1$-morphisms in $\C$ may not be invertible, so this is a feature that $\pi_{\leqslant \infty}(X)$ has but $\C$ does not have.

So we want to find a more general notion of which both $\C$ and $\pi_{\leqslant \infty}(X)$ are particular cases.
This is the notion of an $\infty$-category: a collection of objects and a collection of $n$-morphisms for each $n\in \N_{>0}$ such that composition of $n$-morphisms is associative up to $n+1$-morphism and such that all $n$-morphisms are invertible up to $n+1$-morphism for all $n\in\N_{>1}$.

Besides of being a natural generalization, $\infty$-categories also solve some issues that arise in classical category theory.
The derived category of an abelian category carries a natural triangulated structure, but triangulated categories are not well-behaved with respect to many usual constructions in category theory due to the lack of functoriality of cones.
In particular, despite $\D(\Ab)$ carrying a natural triangulated structure, one should not expect $\Cond(\D(\Ab))$ to be triangulated.
Therefore we should neither expect to have an equivalence $\D(\Cond(\Ab))\cong \Cond(\D(\Ab))$, because even if a triangulation is by definition an extra structure on our category, admitting such an extra structure has very strong implications on the properties of the underlying category, e.g. all monomorphisms and all epimorphisms have to split.



\bibliographystyle{alpha}
\bibliography{refs}

\end{document}
