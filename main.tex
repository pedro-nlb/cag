\documentclass[11pt,A4]{article}
\usepackage[T1]{fontenc}
\usepackage[utf8]{inputenc}
\usepackage[UKenglish]{babel}
\usepackage{mathtools}
\usepackage{amsthm}
\usepackage{amssymb}
%\usepackage{mathrsfs}
\usepackage[mathscr]{euscript}
\usepackage{enumitem}
\usepackage{tikz-cd}
\usepackage{hyperref}
\usepackage[noabbrev]{cleveref}
\usepackage{todonotes}

% Plain theorems
\theoremstyle{plain}
\newtheorem{thm}{Theorem}[section]
\newtheorem{lm}[thm]{Lemma}
\newtheorem{prop}[thm]{Proposition}
\newtheorem{cor}[thm]{Corollary}

% Definition theorems
\theoremstyle{definition}
\newtheorem{defn}[thm]{Definition}
\newtheorem{exa}[thm]{Example}

% Remark theorems
\theoremstyle{remark}
\newtheorem{rem}[thm]{Remark}

% Mathscr for categories
\newcommand{\C}{\mathscr{C}}
\newcommand{\Top}{\mathscr{T}op}
\newcommand{\Ab}{\mathscr{A}b}

% Math operators
\DeclareMathOperator{\Hom}{Hom}
\DeclareMathOperator{\Cond}{Cond}

% New commands
\newcommand{\pe}{*_{pro\acute et}}
\renewcommand{\u}[1]{\underline{#1}}
\newcommand{\ot}{\otimes}
\newcommand{\op}{\oplus}

\title{Seminar on Condensed Mathematics \\ \large Talk 2: Condensed Abelian Groups}
\author{Pedro Núñez}
\date{\today}

\begin{document}

\maketitle

\tableofcontents

\section{The category of condensed abelian groups}

Recall from the previous talk:

\begin{defn}
    The \textit{pro-étale site} of a point, denoted $*_{proét}$, consists of:
    \begin{itemize}
	\item The category whose objects are profinite sets $S$ (topological spaces homeomorphic to an inverse limit of finite discrete topological spaces, a.k.a. totally disconnected compact Hausdorff spaces) and whose morphisms are continuous maps.
	\item The coverings of a profinite set $S$ are all finite families of jointly surjective maps, i.e. all families of morphisms $\{ S_{i}\to S\}_{i\in I}$ indexed by finite sets $J$ such that $\sqcup_{i\in I}S_{i}\to S$ is surjective.
    \end{itemize}
\end{defn}

It is easy to check that the axioms of a covering family are satisfied (see \cite[\href{https://stacks.math.columbia.edu/tag/00VH}{Tag 00VH}]{sta19}), so we have a well-defined site.
Let us start by characterizing sheaves on this site:

\begin{lm}\label{lm:pesheaves}
    A contravariant functor $T$ from $\pe$ to the category of abelian groups $\Ab$ is a sheaf if and only if the following hold:
    \begin{enumerate}[label=\alph*)]
	\item $T(\varnothing)=0$,
	\item $T(S_{1}\sqcup S_{2})\cong T(S_{1})\op T(S_{2})$, and
	\item For any surjection $S'\twoheadrightarrow S$, $T$ induces a group isomorphism
	    \[ T(S)\to \{ g\in T(S') \mid p_{1}^{*}(g)=p_{2}^{*}(g) \in T(S'\times_{S}S') \}, \]
	    where $p_{1},p_{2}\colon S'\times_{S}S'\to S'$ denote the projections.
    \end{enumerate}
    \begin{proof}
	By definition, a functor $T$ is a sheaf on $\pe$ if and only if for every finite family $\{ S_{i}\to S\}_{i\in I}$ of jointly surjective morphisms the diagram
	\[ T(S)\to \prod_{i\in I} T(S_{i})\overset{p_{1}^{*}}{\underset{p_{2}^{*}}{\rightrightarrows}}\prod_{(i,j)\in I^{2}} T(S_{i}\times_{S} S_{j}) \]
	is exact, meaning that the left arrow is an equalizer of the two arrows on the right.
	Since we are in $\Ab$ and $I$ is a finite set, we can reformulate this as the sequence
	\begin{equation}\label{eqn:sheaf}
	    0\to T(S)\to \bigoplus_{i\in I} T(S_{i})\xrightarrow{p_{1}^{*}-p_{2}^{*}} \bigoplus_{(i,j)\in I^{2}} T(S_{i}\times_{S}S_{j})
	\end{equation}
	being exact.

	So assume first that this is the case.
	The empty set is covered by the empty family, so the $T(\varnothing)$ must be a subgroup of $0$, hence $0$ itself.
	This shows $a)$.
	The natural inclusions $S_{i}\to S_{1}\sqcup S_{2}=S$ cover $S$ for $i\in I=\{1,2\}$, so it suffices to show that $p_{1}^{*}=p_{2}^{*}$ to verify $b)$.
	Since $i_{1}(S_{1})\cap i_{2}(S_{2})=\varnothing $, we have $S_{1}\times_{S}S_{2}=S_{2}\times_{S}S_{1}=\varnothing $.
	So if $(g,h)\in T(S_{1})\op T(S_{2})$, then
	\begin{multline*}
	    p_{1}^{*}(g,h)=(T(p_{1}^{S_{1}\times_{S}S_{1}})(g),T(p_{1}^{S_{1}\times_{S}S_{2}})(g),T(p_{1}^{S_{2}\times_{S}S_{1}})(h),T(p_{1}^{S_{2}\times_{S}S_{2}})(h)) \\ 
	    =(T(p_{1}^{S_{1}\times_{S}S_{1}})(g),0,0,T(p_{1}^{S_{2}\times_{S}S_{2}})(h)) \\
	    =(T(p_{2}^{S_{1}\times_{S}S_{1}})(g),T(p_{2}^{S_{1}\times_{S}S_{2}})(h),T(p_{2}^{S_{2}\times_{S}S_{1}})(g),T(p_{2}^{S_{2}\times_{S}S_{2}})(h))=p_{2}^{*}(g,h).
	\end{multline*}
	To show $c)$, note that $S'\twoheadrightarrow S$ is already a cover, so \cref{eqn:sheaf} immediately implies the result.

	Let us now see the converse by induction on the cardinality of $I$.
	If $I=\{1\}$, then we have a surjection $S_{1}\twoheadrightarrow S$ and we are done by $c)$.
	If $I=\{ 1,\ldots, m\}$, then we have $m$ morphisms $f_{i}\colon S_{i}\to S$ such that $\sqcup_{i=1}^{m} S_{i}\twoheadrightarrow S$ is surjective.
	Write $S'=f_{1}(S_{1})\subseteq S$ and $S''=S\setminus S'$.
	By $b)$ we have $T(S)=T(S')\op T(S'')$.
	By $c)$ we have an exact sequence
	\[ 0\to T(S')\to T(S_{1})\to T(S_{1}\times_{S'}S_{1})=T(S_{1}\times_{S}S_{1}), \]
	and since $S''$ is covered by the remaining $m-1$ morphisms, by induction hypothesis we have an exact sequence
	\[ 0\to T(S'')\to \bigoplus_{i=2}^{m}T(S_{i})\to \bigoplus_{(i,j)\in I^{2}\setminus \{(1,1)\}} T(S_{i}\times_{S}S_{j}). \]
	Taking the direct sum of these two exact sequences finishes the proof.
    \end{proof}
\end{lm}

\bibliographystyle{alpha}
\bibliography{refs}

\end{document}
