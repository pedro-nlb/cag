\documentclass[11pt,A4]{article}
\usepackage[T1]{fontenc}
\usepackage[utf8]{inputenc}
\usepackage[UKenglish]{babel}
\usepackage{mathtools}
\usepackage{amsthm}
\usepackage{amssymb}
%\usepackage{mathrsfs}
\usepackage[mathscr]{euscript}
\usepackage{enumitem}
\usepackage{tikz-cd}
\usepackage{hyperref}
\usepackage[noabbrev]{cleveref}
\usepackage{todonotes}

% Plain theorems
\theoremstyle{plain}
\newtheorem{thm}{Theorem}[section]
\newtheorem{lm}[thm]{Lemma}
\newtheorem{prop}[thm]{Proposition}
\newtheorem{cor}[thm]{Corollary}

% Definition theorems
\theoremstyle{definition}
\newtheorem{defn}[thm]{Definition}
\newtheorem{exa}[thm]{Example}

% Remark theorems
\theoremstyle{remark}
\newtheorem{rem}[thm]{Remark}

% Mathscr for categories
\newcommand{\C}{\mathscr{C}}
\newcommand{\Top}{\mathscr{T}op}

% Math operators
\DeclareMathOperator{\Hom}{Hom}

\title{Seminar on Condensed Mathematics \\ \large Talk 2: Condensed Abelian Groups}
\author{Pedro Núñez}
\date{\today}

\begin{document}

\maketitle

\tableofcontents

\section{The category of condensed abelian groups}

Recall from the previous talk:

\begin{defn}
    The \textit{pro-étale site} of a point, denoted $*_{proét}$, consists of:
    \begin{itemize}
	\item The category whose objects are profinite sets $S$ (topological spaces homeomorphic to an inverse limit of finite discrete topological spaces, a.k.a. totally disconnected compact Hausdorff spaces) and whose morphisms are continuous maps.
	\item For a profinite set $S$, its covering sieves are all finite families $\{ S_{j}\to S\}_{j\in J}$ of jointly surjective maps, i.e. $\sqcup_{j\in J}S_{j}\twoheadrightarrow S$ with $J$ a finite indexing set.
    \end{itemize}
\end{defn}

\bibliographystyle{alpha}
\bibliography{refs}

\end{document}
