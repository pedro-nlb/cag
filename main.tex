\documentclass[11pt,A4]{article}
\usepackage[T1]{fontenc}
\usepackage[utf8]{inputenc}
\usepackage[UKenglish]{babel}
\usepackage{mathtools}
\usepackage{amsthm}
\usepackage{amssymb}
%\usepackage{mathrsfs}
\usepackage[mathscr]{euscript}
\usepackage{enumitem}
\usepackage{tikz-cd}
\usepackage{hyperref}
\usepackage[noabbrev]{cleveref}
\usepackage{todonotes}
\usepackage{bbm}

% Plain theorems
\theoremstyle{plain}
\newtheorem{thm}{Theorem}[section]
\newtheorem{lm}[thm]{Lemma}
\newtheorem{prop}[thm]{Proposition}
\newtheorem{cor}[thm]{Corollary}

% Definition theorems
\theoremstyle{definition}
\newtheorem{defn}[thm]{Definition}
\newtheorem{exa}[thm]{Example}

% Remark theorems
\theoremstyle{remark}
\newtheorem{rem}[thm]{Remark}
\newtheorem{q}[thm]{Question}

% Mathbb
\newcommand{\N}{\mathbb{N}}
\newcommand{\Z}{\mathbb{Z}}
\newcommand{\1}{\mathbbm{1}}

% Mathscr for categories
\newcommand{\C}{\mathscr{C}}
\newcommand{\Top}{\mathscr{T}op}
\newcommand{\CHaus}{\mathscr{CH}aus}
\newcommand{\Ab}{\mathscr{A}b}
\newcommand{\Set}{\mathscr{S}et}
\newcommand{\D}{\mathscr{D}}

% Math operators
\DeclareMathOperator{\Hom}{Hom}
\DeclareMathOperator{\Cond}{Cond}
\DeclareMathOperator{\Ob}{Ob}
\DeclareMathOperator{\im}{im}
\DeclareMathOperator{\PSh}{PSh}
\DeclareMathOperator{\Sh}{Sh}

% New commands
\newcommand{\pe}{*_{pro\acute et}}
\renewcommand{\u}[1]{\underline{#1}}
\newcommand{\ot}{\otimes}
\newcommand{\op}{\oplus}
\newcommand{\fp}[1]{\times_{#1}}
\newcommand{\id}{\mathrm{id}}
\newcommand{\ev}{\mathrm{ev}}


\title{Seminar on Condensed Mathematics \\ \large Talk 2: Condensed Abelian Groups}
\author{Pedro Núñez}
\date{\today}

\begin{document}

\maketitle

\tableofcontents

\section{The category of condensed abelian groups}

Recall from the previous talk:

\begin{defn}
    The \textit{pro-étale site} of a point, denoted $*_{proét}$, consists of:
    \begin{itemize}
	\item The category whose objects are profinite sets $S$ (topological spaces homeomorphic to an inverse limit of finite discrete topological spaces, a.k.a. totally disconnected compact Hausdorff spaces) and whose morphisms are continuous maps.
	\item The coverings of a profinite set $S$ are all finite families of jointly surjective maps, i.e. all families of morphisms $\{ S_{i}\to S\}_{i\in I}$ indexed by finite sets $J$ such that $\sqcup_{i\in I}S_{i}\to S$ is surjective.
    \end{itemize}
\end{defn}

It is easy to check that the axioms of a covering family are satisfied (see \cite[\href{https://stacks.math.columbia.edu/tag/00VH}{Tag 00VH}]{sta19}), so we have a well-defined site.
Let us start by characterizing sheaves on this site:

\begin{lm}\label{lm:sheaf}
    A contravariant functor $T$ from $\pe$ to the category of abelian groups $\Ab$ is a sheaf if and only if the following three conditions hold:
    \begin{enumerate}[label=\alph*)]
	\item $T(\varnothing)=0$.
	\item For all profinite sets $S_{1}$ and $S_{2}$, the inclusions $i_{1}\colon S_{1}\to S_{1}\sqcup S_{2}$ and $i_{2}\colon S_{2}\to S_{1}\sqcup S_{2}$ induce a group isomorphism
	    \[ T(S_{1}\sqcup S_{2})\xrightarrow{T(i_{1})\times T(i_{2})} T(S_{1})\op T(S_{2}).\]
	\item Every surjection $f\colon S'\twoheadrightarrow S$ induces a group isomorphism
	    \[ T(S)\xrightarrow{T(f)} \{ g\in T(S') \mid T(p_{1})(g)=T(p_{2})(g) \in T(S'\times_{S}S') \}, \]
	    where $p_{1},p_{2}\colon S'\times_{S}S'\to S'$ denote the projections.
    \end{enumerate}
    \begin{proof}
	By definition, a functor $T$ is a sheaf on $\pe$ if and only if for every finite family $\{ S_{i}\to S\}_{i\in I}$ of jointly surjective morphisms the diagram
	\[ T(S)\to \prod_{i\in I} T(S_{i})\overset{p_{1}^{*}}{\underset{p_{2}^{*}}{\rightrightarrows}}\prod_{(i,j)\in I^{2}} T(S_{i}\times_{S} S_{j}) \]
	is exact, meaning that the left arrow is an equalizer of the two arrows on the right, which are explicitly described below.
	Since we are in $\Ab$ and $I$ is a finite set, we can reformulate this as the sequence
	\begin{equation}\label{eqn:sheaf}
	    0\to T(S)\to \bigoplus_{i\in I} T(S_{i})\xrightarrow{p_{1}^{*}-p_{2}^{*}} \bigoplus_{(i,j)\in I^{2}} T(S_{i}\times_{S}S_{j})
	\end{equation}
	being exact.

	So assume first that \ref{eqn:sheaf} is exact.
	The empty set is covered by the empty family, so the $T(\varnothing)$ must be a subgroup of $0$, hence $0$ itself.
	The natural inclusions $S_{i}\to S_{1}\sqcup S_{2}=S$ cover $S$ for $i\in I=\{1,2\}$, so it suffices to show that $T(p_{1})=T(p_{2})$ to verify $b)$.
	Since $i_{1}(S_{1})\cap i_{2}(S_{2})=\varnothing $, we have $S_{1}\times_{S}S_{2}=S_{2}\times_{S}S_{1}=\varnothing $.
	So if $(g,h)\in T(S_{1})\op T(S_{2})$, then
	\begin{multline*}
	    p_{1}^{*}(g,h)=(T(p_{1}^{S_{1}\times_{S}S_{1}})(g),T(p_{1}^{S_{1}\times_{S}S_{2}})(g),T(p_{1}^{S_{2}\times_{S}S_{1}})(h),T(p_{1}^{S_{2}\times_{S}S_{2}})(h)) \\ 
	    =(T(p_{1}^{S_{1}\times_{S}S_{1}})(g),0,0,T(p_{1}^{S_{2}\times_{S}S_{2}})(h)) \\
	    =(T(p_{2}^{S_{1}\times_{S}S_{1}})(g),T(p_{2}^{S_{1}\times_{S}S_{2}})(h),T(p_{2}^{S_{2}\times_{S}S_{1}})(g),T(p_{2}^{S_{2}\times_{S}S_{2}})(h))=p_{2}^{*}(g,h).
	\end{multline*}
	To show $c)$, note that $S'\twoheadrightarrow S$ is already a cover, so the exactness of \ref{eqn:sheaf} immediately implies the result.

	For the converse, suppose that $\{f_{j}\colon S_{j}\to S\}_{j\in J}$ is a collection of jointly surjective morphisms indexed by $J=\{ 1,\ldots,m\}$.
	We have then a surjection $f\colon S \twoheadrightarrow S$, where $S':=\sqcup_{j=1}^{m}S_{j}$ and $f:=\sqcup_{j=1}^{m}f_{j}$.
	By $c)$ we can write $T(S)=\{ g\in T(S')\mid T(p_{1}^{S'\fp{S}S'})(g)=T(p_{2}^{S'\fp{S}S'})(g)\}$.
	But $S'\fp{S}S'=\sqcup_{(a,b)\in J^{2}} S_{a}\fp{S}S_{b}$ and $p_{\varepsilon }^{S'\fp{S}S'}=\sqcup_{(a,b)\in J^{2}} i_{a}\circ p_{\varepsilon }^{S_{a}\fp{S}S_{b}}$ for $\varepsilon \in \{1,2\}$ and for $i_{a}\colon S_{a}\to S'$ the inclusions, so condition $b)$ allows us to rewrite this as
	\[ T(S)=\bigcap_{(a,b)\in J^{2}} \{ (g_{1},\ldots,g_{m})\in T(S_{1})\op \cdots \op T(S_{m})\mid P_{a,b}(g_{a},g_{b})\}\]
	with $P_{a,b}(g_{a},g_{b}):\equiv T(p_{1}^{S_{a}\fp{S}S_{b}})(g_{a})=T(p_{2}^{S_{a}\fp{S}S_{b}})(g_{b})$, which is what we wanted.	
    \end{proof}
\end{lm}

\begin{defn}
    A \textit{condensed abelian group} is a sheaf of abelian groups on $\pe$.
    We denote the category of condensed abelian groups by $\Cond(\Ab)$.
\end{defn}

From \cref{lm:sheaf} we deduce:

\begin{exa}
    Let $G$ be a topological abelian group. Then the functor $\u{G}=\Hom_{\Top}(-,G)\colon \pe\to \Ab$ is a condensed abelian group.
    Conditions $a)$ and $b)$ in \cref{lm:sheaf} are clear.
    For condition $c)$, let $f\colon S'\twoheadrightarrow S$ be a surjection.
    We want to show that $f^{*}\colon \u{G}(S)\to \u{G}(S')$ induces an isomorphism between the set of continuous maps $g\colon S\to G$ and the set of continous maps $h\colon S'\to G$ such that $h\circ p_{1}=h\circ p_{2}$.
    The image of $f^{*}$ is indeed contained in the set of all such maps, because $f\circ p_{1}=f\circ p_{2}$.
    Moreover, since $f$ is an epimorphism, $f^{*}$ is injective.
    So it only remains to show surjectivity.
    Let $h\colon S'\to G$ be a map such that $h\circ p_{1}=h\circ p_{2}$.
    For each $s\in S$, let $g(s):=h(s')$ for some $s'\in f^{-1}(\{s\})$.
    This is well-defined in $\Set$, because if $f(s')=f(s'')=s$, then $(s',s'')\in S'\times_{S}S'$, and thus we can write
    \[ g(s)=h(s')=h\circ p_{1}(s',s'')=h\circ p_{2}(s',s'')=h(s'').\]
    But in fact it is a morphism in $\Top$.
    Indeed, $S$ carries the quotient topology induced by $f$, so $f^{-1}(g^{-1}(U))=h^{-1}(U)$ being open in $S'$ for all $U$ open in $G$ implies that $g^{-1}(U)$ is open in $S$ for all $U$ open in $G$.
\end{exa}

\subsection{Properties}

A category $\C$ is called \textit{abelian} if it satisfies the following properties:
\begin{enumerate}[label=\roman*)]
    \item There exists a zero object $0\in \Ob(\C)$.
    \item For all $A,B\in \Ob(\C)$, their product $A\xleftarrow{p_{A}} A\times B\xrightarrow{p_{B}} B$ and their coproduct $A\xrightarrow{i_{A}}A\sqcup B\xleftarrow{i_{B}} B$ exist in $\C$ and the canonical morphism $A\sqcup B\xrightarrow{(\id_{A}\times 0) \sqcup (0\times \id_{B})} A\times B$ is an isomorphism\footnote{We identify $A\sqcup B$ with $A\times B$ via this isomorphism and call the result the \textit{direct sum} of $A$ and $B$, denoted $A\op B$.} in $\C$.
    \item Every morphisms has a kernel and a cokernel in $\C$.
    \item Every monomorphism in $\C$ is a kernel and every epimorphism in $\C$ is a cokernel.
\end{enumerate}

In particular, $\C$ is naturally an additive category and all finite limits and colimits exist in $\C$.

Recall also Grothendieck's axioms:
\begin{description}
    \item[(AB3)] All colimits exist.
    \item[(AB3*)] All limits exist.
    \item[(AB4)] (AB3) holds and all coproducts are exact.
    \item[(AB4*)] (AB3*) holds and all products are exact.
    \item[(AB5)] (AB3) holds and filtered colimits are exact.
    \item[(AB6)] (AB3) holds and for any family $\{ I_{j}\}_{j\in J}$ of filtered categories indexed by a set $J$ with functors $F_{j}\colon I_{j}\to \C$ the canonical morphism
	\[ \varinjlim_{(i_{j}\in I_{j})_{j}} \prod_{j\in J}F_{j}(i_{j})\to \prod_{j\in J}\varinjlim_{i_{j}\in I_{j}} F_{j}(i_{j}) \]
	is an isomorphism in $\C$.
\end{description}

\begin{rem}
    To check (AB3) on an abelian category it suffices to show that arbitrary coproducts exist, since we can build any colimit from coproducts and coequalizers, and similarly for the dual statement (AB3*).
    Colimits preserve colimits because the colimit functor is left adjoint to the diagonal functor.
    In particular coproducts are always right exact, so to check (AB4) on an abelian category satisfying (AB3) it suffices to check that coproducts preserve monomorphisms, and similarly for the dual statement (AB4*).
\end{rem}

The main goal of this talk is to show that the category $\Cond(\Ab)$ is an abelian category which verifies all the previous axioms.
Except for axioms (AB4*) and (AB6), this is a general fact which holds on any category of sheaves of abelian groups on a site, but we will still give an explicit proof in our situation.
This will be much easier after we express $\Cond(\Ab)$ as the category of sheaves of abelian groups on a simpler site.

\begin{defn}
    A compact Hausdorff\footnote{If we do not require Hausdorffness, an extremally disconnected space could be very connected, e.g. any irreducible topological space.} topological space is called \textit{extremally disconnected} if the closure of every open set is again open.
\end{defn}

Equivalently, a compact Hausdorff topological space is extremally disconnected if the closures of every pair of disjoint open sets are also disjoint.
Therefore extremally disconnected spaces are totally disconnected compact Hausdorff spaces, hence profinite spaces.
The converse is not true:

\begin{exa}
    In an extremally disconnected space every convergent sequence is eventually constant (see \cite[Theorem 1.3]{gle58}), so the $p$-adic integers are profinite but not extremally disconnected, because $(p^{n})_{n=1}^{\infty}$ converges to $0$.
\end{exa}
    
Extremally disconnected spaces are precisely the projective objects in the category $\CHaus$ of compact Hausdorff spaces (see \cite[Theorem 2.5]{gle58}).
Since pullbacks exist and preserve epimorphisms (i.e. surjective continuous functions\footnote{Urysohn's lemma implies that epimorphisms in the category of compact Hausdorff spaces are precisely surjective continuous functions, which is not true in the whole category of Hausdorff spaces (e.g. the inclusion of the complement of a point in the real line).}) in $\CHaus$, a compact Hausdorff space $S$ is extremally disconnected if and only if any surjection $S'\twoheadrightarrow S$ from a compact Hausdorff space admits a section:
\begin{center}
    \begin{tikzcd}
	& S'\arrow[twoheadrightarrow]{d} \\
	S\arrow[equal]{r}\arrow[dashed]{ur} & S
    \end{tikzcd}
\end{center}

The inclusion functor $U\colon \CHaus\to \Top$ has a left adjoint $\beta\colon \Top\to \CHaus$ called the \textit{Stone-\v{C}ech compactification}.
If $X$ is a discrete topological space, then $\beta X$ is extremally disconnected, because if $S\twoheadrightarrow \beta X$ is a continuous surjection from a compact Hausdorff topological space we may first lift the canonical map $X\to \beta X$ to $S$ and then extend it to a section from $\beta X$ by its universal property.
In particular, every compact Hausdorff space admits a surjection $\beta(X_{disc})\twoheadrightarrow X$ from an extremally disconnected space.
This has the following consequence:

\begin{prop}
    $\Cond(\Ab)$ is equvalent to the category of sheaves of abelian groups on the site of extremally disconnected spaces via the restriction from profinite sets.
    \begin{proof}
	Let $T\in \Cond(\Ab)$ and let $S$ be a profinite set.
	Let $f\colon \tilde{S}\twoheadrightarrow S$ and $g\colon \tilde{\tilde{S}}\twoheadrightarrow \tilde{S}\fp{S}\tilde{S}$ be continuous surjections from extremally disconnected spaces.
	By \cref{lm:sheaf} we have $T(S)=\ker(T(p_{1})-T(p_{2}))$, where $p_{1},p_{2}\colon \tilde{S}\fp{S}\tilde{S}$ denote the projections from the fiber product.
	Since $T(\tilde{S}\fp{S}\tilde{S})\xrightarrow{T(g)} T(\tilde{\tilde{S}})$ is injective, the kernel of $T(\tilde{S})\xrightarrow{T(p_{1})-T(p_{2})} T(\tilde{S}\fp{S}\tilde{S})$ is the same as the kernel of the composition $T(\tilde{S})\xrightarrow{T(p_{1}\circ g)-T(p_{2}\circ g)} T(\tilde{\tilde{S}})$.
	Therefore the value of $T$ at the profinite set $S$ is completely determined by the value of $T$ at the extremally disconnected sets $\tilde{S}$ and $\tilde{\tilde{S}}$.
    \end{proof}
\end{prop}

\begin{cor}\label{cor:sheaf}
    $\Cond(\Ab)$ is equivalent to the category of contravariant functors $T$ from the category of extremally disconnected sets to the category of abelian groups such that:
    \begin{enumerate}[label=\alph*)]
	\item $T(\varnothing)=0$.
	\item For all extremally disconnected sets $S_{1}$ and $S_{2}$, the inclusions $i_{1}\colon S_{1}\to S_{1}\sqcup S_{2}$ and $i_{2}\colon S_{2}\to S_{1}\sqcup S_{2}$ induce a group isomorphism
	    \[ T(S_{1}\sqcup S_{2})\xrightarrow{T(i_{1})\times T(i_{2})} T(S_{1})\op T(S_{2}).\]
    \end{enumerate}
    \begin{proof}
	We need to show that condition $c)$ in \cref{lm:sheaf} is automatically satisfied.
	So let $f\colon S'\twoheadrightarrow S$ be a continuous surjection of extremally disconnected spaces.
	Then we can find a section $\sigma\colon S\to S'$ with $f\circ \sigma =\id_{S}$.
	This implies that $T(\sigma)\circ T(f)=\id_{T(S)}$, so $T(f)$ is injective.
	The image of $T(f)$ is contained in $\{ g\in T(\tilde{S})\mid T(p_{1})(g)=T(p_{2})(g)\}$, because $f\circ p_{1}=f\circ p_{2}$.
	And conversely if $g\in T(\tilde{S})$ is such that $T(p_{1})(g)=T(p_{2})(g)$, then $T((\sigma \circ f)\fp{S}\id_{\tilde{S}})(T(p_{1})(g))=T((\sigma \circ f)\fp{S}\id_{\tilde{S}})(T(p_{2})(g))$, so $T(f)(T(\sigma)(g))=g$ and $g$ is in the image of $T(f)$.
    \end{proof}
\end{cor}

\begin{rem}\label{rem:sumcomponent}
    The following formal consequence of $b)$ will be useful later.
    Let $\eta\colon T_{1}\to T_{2}$ be a morphism in $\Cond(\Ab)$ and let $S_{1}$ and $S_{2}$ be two extremally disconnected sets.
    Then under the isomorphisms of \cref{cor:sheaf} $b)$ the component of $\eta$ at $S_{1}\sqcup S_{2}$ is given by $\eta_{S_{1}}\op \eta_{S_{2}}$, i.e. by
    \[ T_{1}(S_{1})\op T_{1}(S_{1})\xrightarrow{\begin{pmatrix} \eta_{S_{1}} & 0 \\ 0 & \eta_{S_{2}} \end{pmatrix}} T_{2}(S_{1})\op T_{2}(S_{2}).\]
	Indeed, $\eta_{S_{1}}\op \eta_{S_{2}}=(\eta_{S_{1}}\circ p_{T_{1}(S_{1})})\times (\eta_{S_{2}}\circ p_{T_{1}(S_{2})})$, so this follows from the commutativity of the following diagram:
	\begin{center}
	    \begin{tikzcd}
		T_{1}(S_{1})\arrow{r}{\eta_{S_{1}}} & T_{2}(S_{1}) \\
		T_{1}(S_{1}\sqcup S_{2})\arrow{u}{T_{1}(i_{1})}\arrow{r}{\eta_{S_{1}\sqcup S_{2}}}\arrow[swap]{d}{T_{1}(i_{1})\times T_{1}(i_{2})} & T_{2}(S_{1}\sqcup S_{2})\arrow[swap]{u}{T_{2}(i_{1})}\arrow{d}{T_{2}(i_{1})\times T_{2}(i_{2})} \\
		T_{1}(S_{1})\op T_{1}(S_{2})\arrow[bend left=90]{uu}{p_{T_{1}(S_{1})}}\arrow[bend right=90,swap]{dd}{p_{T_{1}(S_{2})}}\arrow[dashed]{r} & T_{2}(S_{1})\op T_{2}(S_{2})\arrow[bend right=90,swap]{uu}{p_{T_{2}(S_{1})}}\arrow[bend left=90]{dd}{p_{T_{2}(S_{2})}} \\
		T_{1}(S_{1}\sqcup S_{2})\arrow{u}{T_{1}(i_{1})\times T_{1}(i_{2})}\arrow{r}{\eta_{S_{1}\sqcup S_{2}}}\arrow[swap]{d}{T_{1}(i_{2})} & T_{2}(S_{1}\sqcup S_{2})\arrow[swap]{u}{T_{2}(i_{1})\times T_{2}(i_{2})}\arrow{d}{T_{2}(i_{2})} \\
		T_{1}(S_{2})\arrow[swap]{r}{\eta_{S_{2}}} & T_{2}(S_{2})
	    \end{tikzcd}
	\end{center}
\end{rem}

With this nice description we are ready to prove the main result of this talk:

\begin{thm}
    $\Cond(\Ab)$ is an abelian category satisfying axioms (AB3), (AB3*), (AB4), (AB4*), (AB5) and (AB6).
    Moreover, it is generated\footnote{A category $\C$ is said to be \textit{generated} by a set $\mathscr{S}\subseteq \Ob(\C)$ if for all pairs of distinct parallel arrows $f\neq g\colon A\rightrightarrows B$ we can find some $h\colon S\to A$ with $S\in \mathscr{S}$ such that $fh\neq gh$.
    As Mac Lane points out, the word \textit{separated} would have been a better choice (see \cite[Section V.7]{mac78}).} by compact\footnote{An object $M$ on an abelian category $\C$ is called \textit{compact} if $\Hom_{\C}(M,-)$ commutes with filtered colimits.} projective objects.
    \begin{proof}
	In \cref{cor:sheaf} we have described $\Cond(\Ab)$ as the category of contravariant functors from extremally disconnected spaces to $\Ab$ sending finite disjoint unions to finite products.
	The idea is to use this description to see that we can construct limits and colimits pointwise.
	Once we have shown that pointwise limits and colimits are sheaves, checking the necessary universal properties and isomorphisms is reduced to checking the corresponding statements in $\Ab$ pointwise, so we will not mention these computations explicitly.
	
	Let us do the case of limits, for example.
	Let $F\colon \mathscr{J}\to \Cond(\Ab)$ be a diagram in $\Cond(\Ab)$, i.e. a functor from some indexing category $\mathscr{J}$.
	For an extremally disconnected set $S$, let $L(S)=\lim(\ev_{S}\circ F)$ in $\Ab$ and for all $\alpha \in \Ob(\mathscr{J})$ denote by $\varepsilon^{\alpha}_{S}\colon L(S)\to F_{\alpha}(S)$ the corresponding canonical map.
	If $f\colon S\to S'$ is a continuous map of extremally disconnected sets, we have a cone to $\ev_{S}\circ F$ given by the compositions $L(S')\to F_{\alpha}(S')\xrightarrow{F_{\alpha}(f)} F_{\alpha}(S)$ for all $\alpha\in\Ob(\mathscr{J})$, so we get a canonical homomorphism to the terminal cone $L(f)\colon L(S')\to L(S)$ making the resulting diagram commute.
	To show that $L$ is a sheaf we use \cref{cor:sheaf}.
	Condition $a)$ is verified because $\ev_{\varnothing}$ is the constantly zero functor.
	For condition $b)$ let $i_{1}\colon S_{1}\to S_{1}\sqcup S_{2}$ and $i_{2}\colon S_{2}\to S_{1}\sqcup S_{2}$ be the inclusion of two extremally disconnected sets in their disjoint union and consider the following diagram:
	\begin{center}
	    \begin{tikzcd}
		&[-50pt] L(S_{1}\sqcup S_{2})\arrow{rrr}{L(i_{1})\times L(i_{2})}\arrow[swap]{dl}{\varepsilon^{\alpha}_{S_{1}\sqcup S_{2}}} &[-50pt] &[-25pt] &[-50pt] L(S_{1})\op L(S_{2})\arrow{dl}{\varepsilon^{\alpha}_{S_{1}}\op \varepsilon^{\alpha}_{S_{2}}}\arrow{ddr}{\varepsilon^{\beta}_{S_{1}}\op \varepsilon^{\beta}_{S_{2}}} &[-50pt] \\
		T_{\alpha}(S_{1}\sqcup S_{2})\arrow[swap, pos=0.6]{rrr}{T_{\alpha}(i_{1})\times T_{\alpha}(i_{2})}\arrow[swap]{drr}{\eta_{S_{1}\sqcup S_{2}}} & & & T_{\alpha}(S_{1})\op T_{\alpha}(S_{2})\arrow[swap]{drr}{\eta_{S_{1}}\op \eta_{S_{2}}} & & \\
		& & T_{\beta}(S_{1}\sqcup S_{2})\ar[from=uul, "\varepsilon^{\beta}_{S_{1}\sqcup S_{2}}", pos=0.3, crossing over]\arrow[swap]{rrr}{T_{\beta}(i_{1})\times T_{\beta}(i_{2})} & & & T_{\beta}(S_{1})\op T_{\beta}(S_{2})%\arrow{d}{p_{T_{\beta}(S_{1})}} %\\
		% & & & & & T_{\beta}(S_{1})
	    \end{tikzcd}
	\end{center}
	Since limits commute with finite direct sums, the right triangle is the cone corresponding to the limit of the direct sum of diagrams.
	The left triangle is by definition the cone corresponding to the limit of the diagram at its base.
	The bottom square commutes by \cref{rem:sumcomponent}, so the universal property of the limit $L(S_{1})\op L(S_{2})$ induces a unique group homomorphism $L(S_{1}\sqcup S_{2})\to L(S_{1})\op L(S_{2})$ making everything commute.
	Since all arrows going right at the bottom are isomorphisms, this unique morphism is a group isomorphism, so it suffices to show that $L(i_{1})\times L(i_{2})$ makes the whole diagram commute.
	Let us see for example that the $\beta$ square commutes.
	By the universal property of the product, it suffices to show commutativity after composing with each projection.
	Composing with the first projection $p_{T_{\beta}(S_{1})}$ we further reduce our problem to the commutativity of the following square:
	\begin{center}
	    \begin{tikzcd}
		L(S_{1}\sqcup S_{2})\arrow{r}{L(i_{1})} \arrow[swap]{d}{\varepsilon^{\beta}_{S_{1}\sqcup S_{2}}} & L(S_{1})\arrow{d}{\varepsilon^{\beta}_{S_{1}}} \\
		T_{\beta}(S_{1}\sqcup S_{2})\arrow{r}{T_{\beta}(i_{1})} & T_{\beta}(S_{1})
	    \end{tikzcd}
	\end{center}
	But this square commutes by construction of $L(i_{1})$, so we are done proving that $L$ is a sheaf.

	Let us see now that $\Cond(\Ab)$ is generated by compact projective objects.
	The forgetful functor $U\colon \Cond(\Ab)\to \Cond(\Set)$ preserves all limits.
	This follows from the pointwise construction of limits in both cases and for the corresponding statement for the forgetful functor $\Ab\to \Set$.
	Both $\Cond(\Ab)$ and $\Cond(\Set)$ satisfy the necessary conditions for the adjoint functor theorem, so we have a left adjoint functor $\Z[-]\colon\Cond(\Set)\to \Cond(\Ab)$.
	This functor attaches to a condensed set $T$ the sheafification of the presheaf which sends an extremally disconnected set $S$ to the free abelian group generated by $T(S)$, hence the notation.
	For an extremally disconnected set $S$ consider the condensed set $\u{S}=\Hom_{\Top}(S,-)$.
	A morphism between two sheaves in $\Cond(\Set)$ is just a natural transformation between them, so by the Yoneda lemma we have a natural bijection $\Hom_{\Cond(\Ab)}(\u{S},U(M))\cong M(S)$ for all condensed abelian groups $M$.
	Combining this with the previous adjunction we obtain natural bijections $\Hom_{\Cond(\Ab)}(\Z[\u{S}],M)\cong M(S)$ for all condensed abelian groups $M$.
	Since limits and colimits are constructed pointwise in $\Cond(\Ab)$, the evaluation functor $\ev_{S}$ commutes with all limits and colimits, which implies by the previous natural bijection that $\Z[\u{S}]$ is both compact and projective.
	The coproduct of compact objects is compact, and the same for projective objects.
	Hence to prove generation by compact projective objects it suffices to find for all condensed abelian group $M$ a surjection $\bigoplus_{\alpha \in \Lambda}\Z[\u{S_{\alpha}}]\twoheadrightarrow M$ for some collection of extremally disconnected sets $\{S_{\alpha}\}_{\alpha\in \Lambda}$.
	
	So let $M\in \Ob(\Cond(\Ab))$.
	By Zorn's lemma, there is a maximal subobject $M'$ of $M$ such that $M'$ admits a surjection $f\colon \bigoplus_{\alpha\in \Lambda}\Z[\u{S_{\alpha}}]\twoheadrightarrow M'$ for some collection of extremally disconnected sets $\{S_{\alpha}\}_{\alpha\in \Lambda}$.
	Suppose $M'\neq M$, i.e. suppose $M/M'\neq 0$.
	Then we can find some extremally disconnected set $S$ such that $(M/M')(S)\neq 0$.
	By Yoneda this means that we can find at least one non-zero morphism $\bar{g}\colon \Z[\u{S}]\to M/M'$, which by projectivity of $\Z[\u{S}]$ can be lifted to a morphism $g\colon \Z[\u{S}]\to M$ such that $\im(g)\not\subseteq M'$.
	Let then $M''$ be the smallest subobject of $M$ containing $M'$ and $\im(g)$, so that $M'\subsetneq M''\subsetneq M$.
	Then we can find a surjection $(\bigoplus_{\alpha\in \Lambda} \Z[\u{S_{\alpha}}])\op \Z[\u{S}]\overset{f\sqcup g}{\twoheadrightarrow} M''$ contradicting maximality of $M'$.
	This implies that $M'=M$ and finishes the proof.
    \end{proof}
\end{thm}

\begin{rem}
    Compact generation in triangulated categories, Brown representability, etc.
\end{rem}

\subsection{Extra structure}

Roughly speaking, a (\textit{symmetric}) \textit{monoidal structure} on a category $\C$ consists of a functor $(-)\ot(-)\colon \C\times \C\to \C$ which turns $\C$ into a (commutative up to natural isomorphism) monoid with associativity and unit up to natural isomorphism (see \cite{nlab:monoidal_category} for the precise definition).
When $\C$ is considered endowed with this extra structure we call it a (\textit{symmetric}) \textit{monoidal category}.

\begin{exa}
    $\Ab$ is a symmetric monoidal category with respect to the usual tensor product and with unit object $\Z$.
\end{exa}

We want to show that $\Cond(\Ab)$ is also a symmetric monoidal category, but for this we need sheafification:

\begin{defn}
    Let $\C$ be a site and let $\Sh(\C)\hookrightarrow \PSh(\C)$ be the inclusion of the category of sheaves to the category of presheaves on $\C$. 
    A \textit{sheafification} functor is a left adjoint $(-)^{a}\colon \PSh(\C)\to \Sh(\C)$ to this inclusion.
\end{defn}

We can explicitly describe the sheafification of a presheaf $F$ as follows.
For a covering family $\mathcal{U}=\{f_{j}\colon U_{j}\to U\}_{j\in J}$ with $J=\{1,\ldots,m\}$ we define the \textit{zero \v{C}ech cohomology} of $F$ with respect to $\mathcal{U}$ as
\[ \check{H}^{0}(\mathcal{U},F)=\bigcap_{(a,b)\in J^{2}} \{ (g_{1},\ldots,g_{m})\in \bigoplus_{j=1}^{m}F(U_{j}) \mid P_{a,b}(g_{a},g_{b})\} \]
with the notation from \cref{lm:sheaf}.
If $\mathcal{U}'\to \mathcal{U}$ is a morphism of covering families of $U$, then we obtain a pullback morphism $\check{H}^{0}(\mathcal{U},F)\to \check{H}^{0}(\mathcal{U}',F)$ between zero \v{C}ech cohomology groups which does not depend on the particular morphism of covering families but only on the covering families themselves.
Define then a presheaf $F^{+}$ by setting
\[ F^{+}(U):=\varinjlim_{\mathcal{U}} \check{H}^{0}(\mathcal{U},F) \]
where the direct limit runs over all covering families of $U$ with $\mathcal{U}\leqslant \mathcal{U'}$ if and only if we have a morphism of covering families $\mathcal{U'}\to \mathcal{U}$ (we can think of $\mathcal{U'}$ being finer $\mathcal{U}$).
This construction takes presheaves into separated presheaves and separated presheaves into sheaves, so taking $F^{a}:=(F^{+})^{+}$ we obtain the desired sheafification.
See \cite[\href{https://stacks.math.columbia.edu/tag/03NQ}{Tag 03NQ}]{sta19} for the precise statements and proofs.

\begin{prop}\label{prop:sm}
    The functor $\ot\colon \Cond(\Ab)\times \Cond(\Ab)\to \Cond(\Ab)$ with $A\ot B:=(S\mapsto A(S)\ot B(S))^{a}$ makes $\Cond(\Ab)$ a symmetric monoidal category with unit given by the constant sheaf $\Z:=(S\mapsto \Z)^{a}$.
\end{prop}

\section{$\infty$-categories}

Let $X$ be a topological space.
We can try to consider this as a category in which the objects are the points of $X$ and the ($1$-)morphisms are the paths in $X$, which can be seen as homotopies between maps $\{ *\}\to X$.
We can consider a homotopy between paths a $2$-morphism, a homotopy between homotopies of paths a $3$-morphism, and so on.
We obtain an object called the \textit{fundamental $\infty$-grupoid} of $X$, denoted $\pi_{\leqslant \infty}(X)$.
Even if we just consider its objects and its $1$-morphisms, $\pi_{\leqslant \infty }(X)$ is not a category, because the composition of paths is not associative.
But it is associative up to a $2$-morphism, and similarly composition of $n$-morphisms is only associative up to an $n+1$-morphism.
Note also that $n$-morphisms are invertible up to an $n+1$-morphism for all $n\in \N_{>0}$.

\begin{q}
    How similar or different is $\pi_{\leqslant \infty}(X)$ from an usual category?
\end{q}

If $\C$ is any category, then we can think of it as having no $n$-morphisms for any $n\in \N_{>1}$.
Then associativity up to higher morphisms is still true in this case, so we can say that this is something that $\C$ and $\pi_{\leqslant \infty }(X)$ have in common.
On the other hand, the $1$-morphisms in $\C$ may not be invertible, so this is a feature that $\pi_{\leqslant \infty}(X)$ has but $\C$ does not have.

So we want to find a more general notion of which both $\C$ and $\pi_{\leqslant \infty}(X)$ are particular cases.
This is the notion of an $\infty$-category: a collection of objects and a collection of $n$-morphisms for each $n\in \N_{>0}$ such that composition of $n$-morphisms is associative up to $n+1$-morphism and such that all $n$-morphisms are invertible up to $n+1$-morphism for all $n\in\N_{>1}$.

Besides of being a natural generalization, $\infty$-categories also solve some issues that arise in classical category theory.
The derived category of an abelian category carries a natural triangulated structure, but triangulated categories are not well-behaved with respect to many usual constructions in category theory due to the lack of functoriality of cones.
In particular, despite $\D(\Ab)$ carrying a natural triangulated structure, one should not expect $\Cond(\D(\Ab))$ to be triangulated.
Therefore we should neither expect to have an equivalence $\D(\Cond(\Ab))\cong \Cond(\D(\Ab))$, because even if a triangulation is by definition an extra structure on our category, admitting such an extra structure has very strong implications on the properties of the underlying category, e.g. all monomorphisms and all epimorphisms have to split.

\section{Comparison with Pyknotic objects}

Barwick and Haine have developed independently a notion very similar to the condensed objects of Clausen and Scholze in their recent paper \cite{bh19}.
This is the notion of \textit{Pyknotic}\footnote{This name comes from the greek word \textit{pykno}, which means dense, compact or thick.} object, which differs from the notion of condensed object on set theoretical matters.



\bibliographystyle{alpha}
\bibliography{refs}

\end{document}
